
\clearpage% Move to first page of new chapter


\chapternonumcenter{摘要}
\chapter*{\centerline{\Title}}

\vspace{10pt}

\begin{center}
    \zihao{-3} \heiti
    摘要
    \\
\end{center}

现有推荐场景下各实体间多具有较强关系约束,为用户量身定制推荐内容已成为推 荐领域重点研究方向。我们以护肤品种类搭配方案为研究背景,着重关注相对顺序之间 有着较强约束关系的物品的推荐,论证研究知识图谱技术在顺序推荐领域的普遍应用方 案。我们利用知识图谱的推理能力和可解释性,结合图谱中存在的约束关系,设计了适 用于这种场景的顺序推荐算法,并对结果进行了验证和可行性分析。

\vspace{12pt}

\noindent \textbf{关键词:} 顺序推荐;知识图谱构建;可解释性;文本分类






\clearpage
\chapternonumcenter{ABSTRACT}



\chapter*{\centerline {A KNOWLEDGE GRAPH­BASED EXPLAINABLE}}
\begin{center}
    \zihao{3}
    RECOMMENDER SYSTEM
    \\
\end{center}

\vspace{12pt}

\begin{center}
    \zihao{-3}
    ABSTRACT
    \\
\end{center}

\vspace{12pt}

This paper presents an empirical exploration of the use 
of capsule networks for text classification. While it has 
been shown that capsule networks are effective for image 
classification, their validity in the domain of text has 
not been explored. In this paper, we show that capsule 
networks in- deed have potential for text classification, 
and that they have several advantages over convolutional 
neural networks. We further suggest a simple routing 
method that ef- fectively reduces the computational 
complexity of dynamic routing. We utilized seven benchmark 
datasets to demonstrate that capsule networks, 
along with the pro- posed routing method provide 
comparable results.

\vspace{20pt}

\noindent \textbf{KEY WORDS:} \textsl{networks; capsules; text classification}
