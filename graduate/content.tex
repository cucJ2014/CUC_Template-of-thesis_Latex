\chapter{绪论}

\section{研究背景及意义}\label{section1}

推荐系统在生活中帮助我们解决了很多问题,通过使用形形色色的APP,我们可以轻松地查看附近的美食、
寻找自己喜欢的音乐、对出行路线进行最优规划等。从推荐系统与深度学习算法结合之前的发展上来看,
可以分为以下几个部分。
为了更好地融合用户和物品的特征、上下文信息而进行有效地推荐,
因子分解机模型通过学习特征的隐权重向量,使得模型泛化能力、解决数据稀疏能力都有了极大地提高\cite{rendle2010factorization}。

\chapter{相关技术}
围绕LaTeX基本用法举例讲解。
\section{编辑}
\subsection{公式编辑}
$Score$函数设计如\autoref{equ:score}。
\begin{equation}
    \label{equ:score}
    {Score}_{list} = B \times \lambda \times \sum_{j}^{list}{\alpha}_{wj} +  C \times
    \eta \times \sum_{j}^{list}{\alpha}_{w'j} + {\beta}_{list} - D \times {\gamma}_{list} 
\end{equation}

\subsection{表格编辑}
\autoref{tab:result}列举了每个模型的平均准确度以及最后的平均准确度。
\begin{table}[h]
    \caption{\label{tab:result}平均准确度}
    \begin{center}
        \begin{tabular}{{p{2cm}<{\centering}|p{2cm}p{2cm}p{2cm}p{2cm}p{2cm}}}% 10cm 減去前兩個欄位寬度後,剩下的通通給
            \hline
            $ k $ & 1 & 2 & 3 & 4 & 5 \\
            \hline
            $ \overline{Acc} $  & 0.8145 & 0.7876 & 0.7851 & 0.7856 & 0.7855 \\
            \hline
            \textbf{Average} & \multicolumn{5}{c}{\textbf{0.7917}}\\ 
            \hline
        \end{tabular}
    \end{center}
\end{table}


\subsection{图片插入与引用}
中国传媒大学新校徽如\autoref{fig:logo}所示。
\begin{figure}[htbp]
    \centering
    \includegraphics[width=.5\linewidth]{/logo/中国传媒大学校徽.png}
    \caption{\label{fig:logo}中国传媒大学校徽}
\end{figure}

\subsection{算法编辑}
综上所述,我们将召回层算法总结在\autoref{alg:Algorithm}中。
\begin{algorithm}  
    \caption{Recall algorithm for our system}
    \label{alg:Algorithm}
    \KwIn{The type of skincare $n_1$,$n_2$,…,$n_k$;
    The price of each type $p_1$,$p_2$,…,$p_k$, $k \in \{1,2,3,4,5,6\}$;
    The skin type, $q \in \{Normal,Oily,Dry,Combination\} \times \{Sensitive, Not Sensitive\}$
    }  
    \KwOut{The set of each recommend list, $S$;}  
    
    select $S_1 = \{s|s_i \in {G}, j.price \in [p-5, p+5] \}, i \in \{n_1,n_2,$…$,n_k\} , j \in s_i$;   
    
    select $S_2 = \{s|s_i \in {S_1}, j.type  \subseteq q $\}; 
    
    $S_3=\{\}$;
    
    \For{$s_i$ in $S_2$}
    {
      Condition 1 : $\exists j \in s_i$ suits for $q$\;
      Condition 2 : $\exists g \in j.ingredients, g$ is in conflict with $q$\;
      \If{! ( Condition 1 and Condition 2 )}
      {
        add $s_i$ to $S_3$\;
      }
    }

    $S_4=\{\}$;

    \For{$s$ in $S_3$} % 遍历类别
    {
       $list = []$ \;
      \For{$i$ in each $k$}  % 遍历类别
      {
        \For{($h=i+1$; $h<k$; $h++$)}  % 前一项类别和后一类别
        {
          \If{(Match($s_{ni}$,$s_{nh}$))} % 每个产品之间是否可达
          {
            add $s_{ni}$,$s_{nh}$ to list\;
          }
        }
      }
      add $list$ to $S_4$\;
    }

    $S_5=\{\}$;

    \For{$list$ in $S_4$}
    {
        \If{(Function(list))}
        {
            add $list$ to $S_5$\;
        }
    }

    \Return $S_5$\;  
\end{algorithm}  

\chapter{总结与展望}
