\chapter{绪论}
在原有的 zjuthesis 模板~\cite{zjuthesis}基础上开发而来。
同学们使用时要注意对照模板与要求,切不可盲目使用。

\section{研究背景及意义}

如果你在Overleaf上编译本模板,请注意如下事项:

\begin{itemize}
    \item 删除根目录的 ``.latexmkrc'' 文件,否则编译失败且不报任何错误
    \item 字体有版权所以本模板不能附带字体,请务必手动上传字体文件,并在各个专业模板下手动指定字体。
        
\end{itemize}


\section{研究现状}

我们可以用includegraphics来插入现有的jpg等格式的图片,
如\autoref{fig:zju-logo}所示。

\begin{figure}[htbp]
    \centering
    \includegraphics[width=.3\linewidth]{logo/zju}
    \caption{\label{fig:zju-logo}大学LOGO}

\end{figure}

\par 如\autoref{tab:sample}所示,这是一张自动调节列宽的表格。

\begin{table}[htbp]
    \caption{\label{tab:sample}自动调节列宽的表格}
    \begin{tabularx}{\linewidth}{c|X<{\centering}}
        \hline
        第一列 & 第二列 \\ \hline
        xxx & xxx \\ \hline
        xxx & xxx \\ \hline
        xxx & xxx \\ \hline
    \end{tabularx}
\end{table}


\par 如\autoref{equ:sample},这是一个公式

\begin{equation}
    \label{equ:sample}
    A=\overbrace{(a+b+c)+\underbrace{i(d+e+f)}_{\text{虚数}}}^{\text{复数}}
\end{equation}

\section{研究内容}

\section{论文结构}

\chapter{相关技术}

\begin{figure}[htbp]
    \centering
    \includegraphics[width=.3\linewidth]{example-image-a}
    \caption{\label{fig:fig-placeholder}图片占位符}
\end{figure}

\section{推荐技术}


\subsection{推荐算法的发展}


\subsection{常用推荐算法}


\section{序列化推荐}


\subsection{序列化推荐算法的定义}


\subsection{序列化推荐算法描述}



\section{可解释性推荐}


\subsection{可解释推荐分类}


\subsection{可解释推荐算法}


\subsection{可解释推荐评估方法}

\section{知识图谱推荐算法}
\subsection{本体技术}
\subsection{基于知识图谱的推荐算法}

\section{本章小结}

\chapter{知识图谱构建}
\begin{figure}[htbp]
    \centering
    \includegraphics[width=.3\linewidth]{example-image-a}
    \caption{\label{fig:fig-placeholder}图片占位符}
\end{figure}
\section{本体设计}
\section{知识图谱的构建}
\subsection{数据集获取}
\subsection{知识抽取}
\subsection{知识补全}
\subsection{知识存储}
\section{本章小结}

\chapter{可解释性推荐系统的设计与实现}
\section{系统结构}
\section{推荐模型}
\subsection{推荐算法描述}
\subsection{推荐算法实现}
\section{可解释模型}
\subsection{可解释算法描述}
\subsection{可解释算法实现}
\section{本章小结}

\chapter{实验结果及分析}
\section{实验环境}
\section{评价指标}
\section{实验与结果分析}
\section{本章小结}

\chapter{总结与展望}
\begin{figure}[htbp]
    \centering
    \includegraphics[width=.3\linewidth]{example-image-a}
    \caption{\label{fig:fig-placeholder}图片占位符}
\end{figure}
\section{总结}
\section{展望}
